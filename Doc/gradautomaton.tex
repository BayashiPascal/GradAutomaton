\section*{Introduction}

GradAutomaton is a C library providing structures and functions to manipulate cellular automaton based on Grad structures.\\

It currently implements the following cellular automaton:\\
\begin{itemize}
\item GradAutomatonWolframOriginal: Cellular automaton described page 53 of "A new kind of science" by S. Wolfram
\item GradAutomatonNeuraNet: Cellular Automaton on GradSquare and GradHexa where the automaton function is a NeuraNet
\end{itemize}

It uses the \begin{ttfamily}PBErr\end{ttfamily}, \begin{ttfamily}Grad\end{ttfamily}, \begin{ttfamily}NeuraNet\end{ttfamily} libraries.\\

\section{Definitions}


\section{Interface}

\begin{scriptsize}
\begin{ttfamily}
\verbatiminput{/home/bayashi/GitHub/GradAutomaton/gradautomaton.h}
\end{ttfamily}
\end{scriptsize}

\section{Code}

\subsection{gradautomaton.c}

\begin{scriptsize}
\begin{ttfamily}
\verbatiminput{/home/bayashi/GitHub/GradAutomaton/gradautomaton.c}
\end{ttfamily}
\end{scriptsize}

\subsection{gradautomaton-inline.c}

\begin{scriptsize}
\begin{ttfamily}
\verbatiminput{/home/bayashi/GitHub/GradAutomaton/gradautomaton-inline.c}
\end{ttfamily}
\end{scriptsize}

\section{Makefile}

\begin{scriptsize}
\begin{ttfamily}
\verbatiminput{/home/bayashi/GitHub/GradAutomaton/Makefile}
\end{ttfamily}
\end{scriptsize}

\section{Unit tests}

\begin{scriptsize}
\begin{ttfamily}
\verbatiminput{/home/bayashi/GitHub/GradAutomaton/main.c}
\end{ttfamily}
\end{scriptsize}

\section{Unit tests output}

unitTestRef.txt:
\begin{scriptsize}
\begin{ttfamily}
\verbatiminput{/home/bayashi/GitHub/GradAutomaton/unitTestRef.txt}
\end{ttfamily}
\end{scriptsize}

unitTestGradAutomatonWolframOriginalSave.json
\begin{scriptsize}
\begin{ttfamily}
\verbatiminput{/home/bayashi/GitHub/GradAutomaton/unitTestGradAutomatonWolframOriginalSave.json}
\end{ttfamily}
\end{scriptsize}



